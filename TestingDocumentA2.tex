\documentclass{article}
\usepackage{color}
\usepackage[ocgcolorlinks]{hyperref} %ocg coverts links to black when you print -- downside of unbreakable lines

%%% Package: Listings
\usepackage{listings} % for formatting code examples
\lstset{breaklines=true % sets automatic line breaking
        ,numbers=left % where to put the line-numbers; possible values are (none, left, right)
        ,numbersep=5pt % how far the line-numbers are from the code
        ,numberstyle=\tiny\color[rgb]{0.5,0.5,0.5} % the style that is used for the line-numbers
        ,keepspaces=true                 % keeps spaces in text, useful for keeping indentation of code (possibly needs columns=flexible)
        ,breakatwhitespace=false
        ,showlines=true
        ,numberblanklines=true
        ,frame=single
        }
%%%
  
\title{\bf Testing Document}
\date{\vspace{-5ex}}

\begin{document}
\maketitle
\tableofcontents

\section{Introduction}
This document will outline testing procedures for the Checkers game. When first starting the game, the user is shown a menu with two(2) options - both of which correspond to a method of board setup 
\begin{enumerate}
\item PLAY
\item CUSTOM
\item LOAD
\end{enumerate}

\section{PLAY}
When the user clicks on PLAY, a standard 8x8 checkers board with all 24 pieces (12 white, 12 black) should be generated.
To test this functionality click on the play button and verify the correct board has been generated.

\section{CUSTOM}
When the user clicks on CUSTOM, they will be prompted to enter positions for all of their pieces.

\subsection{No Input}
In the event that the user inputs nothing, the console will display an incorrect input message and prompt the user for a correct input

\subsubsection*{Test Cases}
\begin{itemize}
\item Return Key
\end{itemize}

\subsection{Input Incorrectly Formatted}
In the event that the user inputs the piece information incorrectly, the console will display an appropriate incorrect input message and prompt the user for a correct input. 

\subsubsection*{Test Cases}
\begin{itemize}
\item c
\item A1=e
\item W=A1
\item A1 = W
\item A1 = KW
\item 2=2
\item =====
\item !@**
\item A(5-4)=W
\item AA3=W
\item A1=W, C1=B
\item A1=W,\qquad C1=B
\item G1=34,A7=B
\end{itemize}

\subsection{Invalid Location - Not on Solid Square}
In the event that the user inputs a location that corresponds to a light square instead of a solid square, the console will display an appropriate incorrect input message and prompt the user for a correct input

\subsubsection*{Test Cases}
\begin{itemize}
\item A2=W
\item B1=B
\item C4=W
\item F5=W
\item H3=B
\end{itemize}

\subsection{Invalid Location - Out Of Board Bounds}
In the event that the user inputs a location that does not exist on the board, the console will display an incorrect input message and prompt the user for a correct input.

\subsubsection*{Test Cases}
\begin{itemize}
\item A9=B
\item B12=W
\item I1=W
\item J10=B
\end{itemize}

\subsection{Too Many White Pieces}
In the event that the user inputs too many white pieces ($>$12), the console will display an appropriate input message (along with how many pieces you inputted) and prompt the user for a correct input.

\subsubsection*{Test Cases}
\begin{itemize}
\item A1=W,C1=W,E1=W,G1=W,A3=W,A5=W,A7=W,B8=W,B6=W,B4=W,
B2=W,E1=W,E3=W
\item E5=B,A1=W,C1=W,E1=W,G1=W,A3=W,A5=W,A7=W,B8=W,B6=W,
B4=W,B2=W,E1=W,E3=W
\end{itemize}

\subsection{Too Many Black Pieces}
In the event that the user inputs too many black pieces ($>$12), the console will display an appropriate input message (along with how many pieces you inputted) and prompt the user for a correct input.

\subsubsection*{Test Cases}
\begin{itemize}
\item A1=B,C1=B,E1=B,G1=B,A3=B,A5=B,A7=B,B8=B,B6=B,B4=B, \\
B2=B,E1=B,E3=B
\item A1=B,C1=B,E1=B,G1=B,A3=B,A5=B,A7=B,B8=B,B6=B,B4=B, \\
B2=B,E1=B,E3=B,E5=W
\end{itemize}

\subsection{Overlapping Pieces}
In the event that the user inputs a location that is already filled with a piece, the previous piece will be overwritten by the new piece
\subsubsection*{Test Cases}
\begin{itemize}
\item A1=B,A1=W
\item A1=W,A1=WK
\item A1=W,A3=B,A1=B,A3=W
\item A1=W,A1=W
\end{itemize}

\subsection{Accepted Board Configurations}
In the event that the user inputs the correct format, the console will display a confirmation message and the custom game board will be generated 

\subsubsection*{Test Cases}
\begin{itemize}
\item A1=W
\item a1=w
\item A3=W,B2=B
\end{itemize}

\section{SAVE}
The user can press the save button at any time while playing. Currently, the save file is saved into a folder on the desktop.
If the user doesn't have write permissions to the desktop, a message will appear saying "Save Unsuccessful".

\section{LOAD}
When the user clicks on LOAD, if the savefile exists, the game will switch to the playing state with the board set up.
Loading of board configurations is handled internally by the same system as SETUP so the same limitations are shared,
such as: the user cannot load a board with more than 12 pieces for each player.

\subsection{Save File}
If the save file exists, but the text inside doesn't hold legal parsable information, the game will stay in the menu and tells the user a save cannot be found.
Only the first two lines of the save file are parsed. The first line contains who's turn it is to go out of {BLACK, WHITE}.
And the second line contains the board setup as a string in the same format as inputted by a user.
The user should not be writing in this file manually.

\subsubsection*{Test Cases}
\begin{itemize}
\item Default board\hfill
\begin{lstlisting}
BLACK
A1=W,A3=W,A7=B,B2=W,B6=B,B8=B,C1=W,C3=W,C7=B,D2=W,D6=B,D8=B,E1=W,E3=W,E7=B,F2=W,F6=B,F8=B,G1=W,G3=W,G7=B,H2=W,H6=B,H8=B"
\end{lstlisting}

\item Custom board\hfill 
\begin{lstlisting}
BLACK
A1=W,A3=W,A7=B,B4=B,B8=B,C1=BK,C5=B,C7=W,D4=BK,D6=W,E3=W,F4=B,F6=B,F8=WK,G1=W,G3=W,G5=W,G7=WK,H6=B
\end{lstlisting}

\item Empty board\hfill 
\begin{lstlisting}
blAcK

\end{lstlisting}

\item Extra information\hfill 
\begin{lstlisting}
WHITE
A1=W,A3=W,A7=B,B4=B,B8=B,C1=BK,C5=B,C7=W,D4=BK,D6=W,E3=W,F4=B,F6=B,F8=WK,G1=W,G3=W,G5=W,G7=WK,H6=B
extra information and junk here
is allowed in case we want to save score and such later on

\end{lstlisting}
\end{itemize}

\end{document}